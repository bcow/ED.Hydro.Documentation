\newpage
\addcontentsline{toc}{section}{Background}
\section*{Background}

\phantomsection
\addcontentsline{toc}{subsection}{Tropical forests under climate change}
\subsection*{Tropical forests under climate change}

Tropical forests play an integral part in the terrestrial carbon cycle. The terrestrial biosphere is responsible for the uptake of 29\% of \ch{CO2} emissions \citep{friedlingstein_2019}, accounting for approximately 45\% of terrestrial carbon stocks \citep{bonan_2008}. Tropical forests comprise the highest percentage area of global forest biomes \citep{pan_2011} and the Amazon rainforest, the largest continuous tropical forest in the world, \citep{malhi_2008} alone stores an estimated 120 billion tons of carbon \citep{phillips_2009} and accounts for 15\% of terrestrial productivity \citep{field_1998}.

However, under climate change, the role of tropical forests in the terrestrial carbon cycle remains unclear. Climate change projections include predictions of increased variability in global climate conditions, resulting in increased frequency of extreme events such as early and late heat waves, droughts and floods \citep{ipcc_2015}.  A range of global climate model predictions (GCMs) have predicted significant shifts in temperature and precipitation patterns for the tropics \citep{maloney_2014}. In the Amazon region, a range of GCM projections have been shown to predict increased intensity of wet and dry seasons \citep{lintner_2012} as well as extended dry season length \citep{malhi_2009, malhi_2008}, effectively increasing both the frequency and amplitude of stressful conditions for tropical forests.

The combination of reduced precipitation and subsequent reduction in soil moisture with increased temperatures and evaporative demand from the atmosphere causes plant water stress \citep{choat_2018}, one of the main drivers of tree mortality in tropical forests \citep{rowland_2015,mcdowell_2018}. Studies from precipitation throughfall experiments \citep{nepstad_2007} \todoc{Meir, 2015} as well as  naturally occurring drought events \citep{lewis_2011,phillips_2009} have shown that tropical forest are vulnerable to water stress and the effects of long term stress could have adverse impacts on the global carbon cycle. Therefore, to confidently predict the state of the future terrestrial carbon sink, we must first be able to predict tropical forest responses to climate change.

\addcontentsline{toc}{subsection}{Modeling Plant Response to Water Stress}
\subsection*{Modeling Plant Response to Water Stress}

All plants are subject to the fundamental tradeoff between obtaining carbon and losing water \citep{bonan_2014}. \todoc{New Bonan book?} However, our understanding of the mechanisms that govern species responses to water stress are still limited. Traditionally terrestrial biosphere models (TBMs) represent water stress driven by (1) atmospheric water demand and (2) root water supply.

\paragraph{Atmospheric demand}

The majority of TBMs use variants of the Ball-Berry model: a coupled relationship between stomatal conductance and \ch{CO2} assimilation that varies with atmospheric (or leaf-surface) \ch{CO2} concentration, and some measure of atmospheric humidity \citep{farquhar_1984, ball_1987, leuning_1995, medlyn_2011}.

\paragraph{Soil moisture supply}

To account for the effect of root water supply, photosynthesis is downregulated via an empirical factor (often denoted $\beta$) driven by soil moisture availability \citep{rogers_2017}. Specific formulation of the $\beta-$factor response to soil water content differs between models, but the relation remains empirical, and thus a large contributor to predictive uncertainty across models. 
 
Models formulated in this way have been shown to reasonably predict ecosystem carbon fluxes in the tropics under present climate conditions, however, they fail to do  so under extreme drought scenarios \citep{powell_2013, zhang_2015}. A model intercomparison project testing the ability of 10 TBMs to reproduce the observed sensitivity of ecosystem productivity to reduced precipitation found that the modelled sensitivities of vegetation dynamics to be highly uncertain. Furthermore, predictions of vegetation productivity at rainfall exclusion experiments that did agree did not appear to arrive at the same results for for the same reason \citep{paschalis_2020}.

With no ability to constrain the empirical formulation of water stress, the feasibility of using these models to predict tropical forest response to extreme climate events is low.  

% A myriad of parameters such as soil properties, and root traits affect how models extract water from the soil, which in turn affects plant water status and how plants change the soil moisture at different temporal scales.


\addcontentsline{toc}{subsubsection}{Mechanistic Hydraulic models}
\subsection*{Mechanistic Hydraulic models}

A growing body of research recommends mechanistic hydraulic models based on cohesion-tension theory to model plant water stress \citep{rogers_2017, fisher_2018, choat_2018}. These models link stomatal function with soil water potential, tracking water flow along the soil-plant-atmosphere continuum based on hydraulic laws \citep{sperry_2015, williams_1996}. Unlike the empirical $\beta-$factor approach, hydraulic model parameters can be derived directly from empirical measurements \citep{choat_2018}.

Implementations of mechanistic hydaulic schemes in TBMs have improved predictive performance at tropical sites.
\todoa{Elaborate - specifically what are the models getting right?}.
ED-hydro: \citep{xu_2016, powell_2018, longo_2019}

\todoc{Missing Longo 2019 Part 2}

\todoq{I could expand to other model specific citations I haven't found a review paper about benchmarking hydraulic models (ie the hydraulic version of \citep{paschalis_2020})

* \citep{fisher_2007}  SPA \\
* Christofersen 2016  FATES (also Koven 2020) \\
* \citep{fauset_2019}  Trait-based Forest Simulator (TFS) \\
* Kennedy 2019  CLM 
}

\todoc{Christofersen 2016}
\todoc{Kennedy 2019}
\todoc{Koven 2020}

\addcontentsline{toc}{subsection}{The Downsides of Model Complexity}
\subsection*{The Downsides of Model Complexity}

In most cases it is impossible to know the exact values for the  parameters that are used in TBMs. 

\todoq{This seems obvious, but I've worded it poorly. If it were possible to know the exact value then wouldn't it just technically be a constant and not a parameter?
By definition, parameter should have probability density functions. Is that true? Is there a citation for that? }

% given that they are based on either empirical values or derived from observations (CITE?)

Parameter uncertainty propagates through a model and has the potential to contribute a significant amount to the total model predictive uncertainty. It is therefore essential to quantify the uncertainty surrounding parameter estimates to properly quantify total model predictive uncertainty \citep{dietze_2017, dietze_2017a}. 

Mechanistic hydraulic models bring with them a substantial increase in the number of processes, parameters and subsequently, predictive uncertainty. Furthermore, the sensitivity of a mechanistic hydraulic models to their parameters may not be the same under all climactic conditions. It is most likely that the models will have increased sensitivity to hydraulic parameters under water limiting conditions, which would result in increased predictive uncertainty as well.

\todoc{NEEDS TO REFERENCE \citep{fisher_2020} which has a section addressing model complexity!!!}

Unfortunately, hydraulic trait databases are more sparse than for other areas such as foliar traits and collection can be challenging and expensive \citep{choat_2018}. Consequently, there does not currently exist a large amount of data to constrain hydraulic model parameter estimates. \todoc{Maybe Felicien meta analysis?} Thus, despite the potential that mechanistic hydraulic models have to improve predictive performance, there is an inherent risk of increased predictive uncertainty as a trade off.