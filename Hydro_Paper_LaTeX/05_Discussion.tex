\newpage
\addcontentsline{toc}{section}{Discussion}
\section*{Discussion}


When taking in to account total predictive variance, ED2-hydro does not show a clear improvement over ED2, however, this is not due to parameter uncertainty attributed to the addition of hydraulic parameters, but rather the changes in model sensitivity to existing parameters in the model. 

Hydraulic parameters have clear biophysical meaning and thus it was possible to collect data to reduce ED2-hydro model uncertainty related to water uptake and transport. This was not the case with ED2, where hydraulic parameter uncertainty and subsequent model sensitivity to the empirical water conductance parameter uncertainty was very high. Thus, the synthesis of current hydraulic data is already enough to mitigate the effect of adding parameters to ED2-hydro. 

However, the problem lies in ED2-hydro's increased sensitivity to root biomass allocation parameters. While these parameters can be measured in the field, we were not able to collect enough data to constrain parameter uncertainty. 

Something about a model data loop . 
The meta-analysis for the paper was completed in this iterative fashion, using variance decomposition to inform the collection of trait data. Early in the study, sensitivity analyses showed the data constraint of the hydraulic traits was sufficient, revealing the model to be sensitive to fine root allocation which in turn led to ingesting data from FRED. Next ED2-hydro was most sensitive to leaf biomass allometry parameters which led to ingesting data BAAD. This iterative cycle can continue ad infinitum, and we leave it here with the recommendation to collect data root biomass allocation. 

Here we show a successful data constraint project, taking an iterative model-data feedback approach. 


% \clearpage
% \phantomsection
% \addcontentsline{toc}{subsection}{Unaddressed issues}
% \subsection*{Unaddressed Issues}
Unaddressed issues 
\begin{itemize}
    \item The issue with the parameters uncertainty not really being carried through
\end{itemize}