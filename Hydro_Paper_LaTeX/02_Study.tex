\addcontentsline{toc}{section}{Study}
\section*{Study}

Our goal is to understand the tradeoffs in predictive confidence when choosing to use a mechanistic hydraulic model to predict tropical forest responses to projected climate change. We compare the performance of two implementations of the Ecosystem Demography model version 2.2 (ED2), an individual-based terrestrial biosphere model that simulates fast  time-scale  vegetation dynamics using integrated submodels of plant growth and mortality, phenology, disturbance, biodiversity, hydrology, and soil biogeochemistry. The original implementation of ED2 (ED2-orig) uses a simple, non-physical, empirical scheme for regulating plant water use through the ratio of transpirational demand and water supply \citep{moorcroft_2001, medvigy_2009, longo_2019}. ED2 with hydrodynamics (ED2-hydro) second employs a fully mechanistic plant hydraulics module \citep{medvigy_2009, powell_2018, xu_2016, longo_2019}. By  propagating parameter uncertainty of the hydraulic parameters through the models, we were able to assess the predictive skill of the competing models and quantify the changes in uncertainty surrounding their respective predictions.

This enables us to answer the following questions: (1) Does the model refinement of the mechanistic hydraulic model outweigh the potential increases in predictive uncertainty due to increased parameter uncertainty? Do these results change under wet or dry climate conditions? (2) Given the current hydraulic trait data that has been collected, can we sufficiently constrain parametric uncertainty to curtail excessive predictive uncertainty? Do these results change under wet or dry climate conditions? (3) If it is not currently possible to constrain predictive uncertainty, can we use variance decomposition analysis to guide us towards future data collection strategies to ultimately achieve the desired results?