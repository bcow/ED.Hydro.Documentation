\addcontentsline{toc}{section}{Study}
\section*{Study}

Our goal was to account for and explain the changes in model predictive variability when integrating a complex mechanistic hydraulic module in to an existing vegetation demography model. We compared the performance of two implementations of the Ecosystem Demography model version 2.2 (ED2), an individual-based TBM. The original implementation of ED2 (ED2-orig) uses a simple, empirical scheme for regulating plant water use through the ratio of transpirational demand and water supply \citep{moorcroft_2001, medvigy_2009, xu_2016, longo_2019}. ED2 with hydrodynamics (ED2-hydro) employs a fully mechanistic plant hydraulics module \citep{medvigy_2009, powell_2018, xu_2016, longo_2019}. 

By propagating parameter uncertainty attributed to hydraulic parameters through the models, we were able to compare the relative contribution hydraulic parameter uncertainty made to total predictive variance. This enables us to answer the following questions: 

\begin{enumerate}
    \item Does the model refinement of the mechanistic hydraulic model outweigh the potential increases in predictive uncertainty due to increased parameter uncertainty? Do these results change under wet or dry climate conditions?
    \item Given the current hydraulic trait data that has been collected, can we sufficiently constrain parametric uncertainty to curtail excessive predictive uncertainty? Do these results change under wet or dry climate conditions?
    \item If it is not currently possible to constrain predictive uncertainty, can we use variance decomposition analysis to guide us towards future data collection strategies to ultimately achieve the desired results?
\end{enumerate}