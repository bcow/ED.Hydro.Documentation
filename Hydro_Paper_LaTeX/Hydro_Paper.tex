\documentclass[10pt]{article}

%%%%%%%%%%%%%%%%%%%%%%%%%%%%%%%%%%
% Packages
%%%%%%%%%%%%%%%%%%%%%%%%%%%%%%%%%%
\usepackage{latexsym, amssymb, amsmath}
\usepackage{chemformula} % for chemistry

\usepackage{graphicx} % for importing images
\usepackage{wrapfig}  % for wrapping figures

% setup bibliography
\usepackage[round, comma, authoryear, sort]{natbib}

\usepackage{authblk} % for setting author affiliations

% \usepackage{Wiley-AuthoringTemplate}

%% Formatting Page margins

\usepackage{flowchart}
\usepackage[paperheight=11.0in,paperwidth=8.5in,left=1.0in,right=1.0in,top=1.0in,bottom=1.0in,headheight=1in]{geometry}

%% Set Depths for Nested Lists created by \begin{enumerate}

\usepackage{enumitem} % for customizing lists
\setitemize{noitemsep,topsep=0pt,parsep=0pt,partopsep=0pt}

\setlength{\parskip}{\baselineskip} % Vertical space between paragraphs
% \baselineskip - normal vertical distance between lines in a paragraph
% \baselinestretch - multiplies \baselineskip
\setlength{\parindent}{0pt} % Paragraph indentation

\usepackage{titlesec}
\titlespacing\section{0pt}{10pt plus 4pt minus 2pt}{0pt plus 2pt minus 2pt}
\titlespacing\subsection{0pt}{10pt plus 4pt minus 2pt}{0pt plus 2pt minus 2pt}
\titlespacing\subsubsection{0pt}{10pt plus 4pt minus 2pt}{0pt plus 2pt minus 2pt}

%% Something about tifs and pngs???

\DeclareGraphicsRule{.tif}{png}{.png}{`convert #1 `dirname #1`/`basename #1 .tif`.png}

%%%%%%%%%%%%%%%%%%%%%%%%%%%%%%%%%%
%% User Defined Commands 
%%%%%%%%%%%%%%%%%%%%%%%%%%%%%%%%%%
% None for now 

%%%%%%%%%%%
% \begin{acronyms}
%     \acro{ED2}{Ecosystem Demography Model version 2.2}
%     \acro{TBM}{Terrestrial Biosphere Model}
% \end{acronyms}

\usepackage{hyperref} % for customizing reference links
%\hypersetup{
%    colorlinks=true,
%    linkcolor=blue,
%    filecolor=magenta,      
%    urlcolor=blue,
%}
\urlstyle{same}

\begin{document}

%%%%%%%%%%%%%%%%%%%%%%%%%%%%%%%%%%
\title{The Downside of Detail:\\
Comparing Predictive Uncertainty Between Water Transport Models}
    
\author[1,2]{Betsy Cowdery}
% \author[1]{Author B}
% \author[1]{Author C}
% \author[2]{Author D}
% \author[2]{Author E}
% \affil[1]{Department of Computer Science, \LaTeX\ University}
% \affil[2]{Department of Mechanical Engineering, \LaTeX\ University}
% \setcounter{Maxaffil}{0}
% \date{\today}

%%%%%%%%%%%%%%%%%%%%%%%%%%%%%%%%%%

\maketitle

% \newpage

%%%%%%%%%%%%%%%%%%%%%%
\section*{Background}

\subsection*{Tropical forests under climate change}
Tropical forests play an integral part in the terrestrial carbon cycle. The terrestrial biosphere is responsible for the uptake of 29\% of \ch{CO2} emissions \citep{friedlingstein_2019}, accounting for approximately 45\% of the terrestrial carbon stocks \citep{bonan_2008}. Tropical forests comprise the largest area of global forest biomes \citep{pan_2011} and the Amazon rainforest, the largest continuous tropical forest in the world, \citep{malhi_2008} alone stores an estimated 120 billion tons of carbon \citep{phillips_2009} and accounts for 15\% of terrestrial productivity \citep{field_1998}.

However, under climate change, the role of tropical forests in the terrestrial carbon cycle remains unclear. Climate change projections include predictions of increased variability in global climate conditions, resulting in increased frequency of extreme events such as early and late heat waves, droughts and floods \citep{ipcc_2015}.  A range of global climate model predictions (GCMs) have predicted significant shifts in temperature and precipitation patterns for the tropics \citep{maloney_2014}. In the Amazon region, a range of GCM projections predict increased intensity of wet and dry seasons \citep{lintner_2012} as well as extended dry season length \citep{malhi_2009, malhi_2008}.

The combination of reduced precipitation and subsequent reduction in soil moisture with increased temperatures and evaporative demand from the atmosphere causes plant water stress \citep{choat_2018}, one of the main drivers of tree mortality in tropical forests \citep{rowland_2015,mcdowell_2018}. Studies from precipitation throughfall experiments \citep{nepstad_2007} (MEIR 2015) as well as  naturally occurring drought events \citep{lewis_2011,phillips_2009} have shown that tropical forest are vulnerable to water stress and under long term extremes could have adverse impacts on the global carbon cycle. Therefore, to accurately predict the state of the future terrestrial carbon sink with constrained predictive uncertainty, we must first be able to predict tropical forest responses to climate change.

\subsection*{Modeling Plant Response to Water Stress}

All plants are subject to the fundamental tradeoff between obtaining carbon and losing water \citep{bonan_2014}. However, our understanding of the mechanisms that govern species responses to water stress are still limited. Traditionally terrestrial biosphere models (TBMs) represent water stress driven by (1) atmospheric water demand and (2) root water supply.

\subsubsection*{Atmospheric demand}

The majority of TBMs use variants of the Ball-Berry model: a coupled relationship of stomatal conductance with \ch{CO2} assimilation that varies with atmospheric (or leaf-surface) \ch{CO2} concentration, and some measure of atmospheric humidity (Farquhar 1980, Ball 1987, Leuning 1995, Medlyn 2011). 

\subsubsection*{Soil moisture supply}

To account for the effect of root water supply, photosynthesis is downregulated via an empirical factor (often denoted $\beta$) driven by soil moisture availability \citep{rogers_2017}. Specific formulation of the $\beta-$factor response to soil water content differs between models, but the relation remains empirical, and thus a large contributor to predictive uncertainty across models.
 
Models formulated in this way have been shown to reasonably predict ecosystem carbon fluxes under present climate conditions, however, but they fail to do so under extreme drought scenarios \citep{powell_2013, zhang_2015, paschalis_2020}.

\subsubsection*{Hydraulic models}

A growing body of research recommend mechanistic hydraulic models based on cohesion-tension theory to model plant water stress \citep{rogers_2017,fisher_2018}. These models link stomatal function with soil water potential, tracking water flow along the soil-plant-atmosphere continuum based on hydraulic laws \citep{sperry_2015, williams_1996}. Unlike the empirical $\beta-$factor approach, hydraulic model parameters can be derived directly from empirical measurements \citep{choat_2018}.

Implementations of mechanistic hydaulic schemes in TBMs have improved predictive performance at tropical sites (need to expand on these citations.) 
\citep{xu_2016, powell_2018, longo_2019, fisher_2007, fauset_2019, choat_2018}.

\subsection*{The Downsides of Increased Model Complexity}

In most cases it is impossible to know the exact values for the  parameters that are used in TBMs given that they are based on either empirical values or observations (CITE?). Parameter uncertainty propagates through a model and has the potential to contribute a significant amount to the total model predictive uncertainty. It is therefore essential to quantify the uncertainty surrounding parameter estimates to properly quantify total model predictive uncertainty \citep{dietze_2017, dietze_2017a}.

Mechanistic hydraulic models bring with them a substantial increase in the number of processes, parameters and subsequently, predictive uncertainty. Furthermore, the sensitivity of a mechanistic hydraulic models to their parameters may not be the same under all climactic conditions. It is most likely that the models will have increased sensitivity to hydraulic parameters under water limiting conditions, which would result in increased predictive uncertainty as well.

Unfortunately, hydraulic trait databases are more sparse than for other areas such as foliar traits and collection can be challenging and expensive \citep{choat_2018}. Consequently, there does not currently exist a large amount of data to constrain hydraulic model parameter estimates. Thus, despite the potential that the mechanistic hydraulic models have for better predictive performance, there is an inherent risk of increased predictive uncertainty as a trade off.

\section*{Study}

Our goal is to understand the tradeoffs in predictive confidence when choosing to use a mechanistic hydraulic model to predict tropical forest responses to projected climate change. We compare the performance of two implementations of the Ecosystem Demography model version 2.2 (ED2), an individual-based terrestrial biosphere model that simulates fast  time-scale  vegetation dynamics using integrated submodels of plant growth and mortality, phenology, disturbance, biodiversity, hydrology, and soil biogeochemistry. The original implementation of ED2 (ED2-orig) uses a simple, non-physical, empirical scheme for regulating plant water use through the ratio of transpirational demand and water supply \citep{moorcroft_2001, medvigy_2009, longo_2019}. ED2 with hydrodynamics (ED2-hydro) second employs a fully mechanistic plant hydraulics module \citep{medvigy_2009, powell_2018, xu_2016, longo_2019}. By  propagating parameter uncertainty of the hydraulic parameters through the models, we were able to assess the predictive skill of the competing models and quantify the changes in uncertainty surrounding their respective predictions.

This enables us to answer the following questions: (1) Does the model refinement of the mechanistic hydraulic model outweigh the potential increases in predictive uncertainty due to increased parameter uncertainty? Do these results change under wet or dry climate conditions? (2) Given the current hydraulic trait data that has been collected, can we sufficiently constrain parametric uncertainty to curtail excessive predictive uncertainty? Do these results change under wet or dry climate conditions? (3) If it is not currently possible to constrain predictive uncertainty, can we use variance decomposition analysis to guide us towards future data collection strategies to ultimately achieve the desired results?

\section*{Materials and Methods}

\subsection*{Study Site}

\begin{itemize}
    \item 50-ha long term forest monitoring plot
    \item  central Panama
    \item (lat, lon)
    \item moist lowland tropical predominanly evergreen forest, only 10\% of canopy drops leaves during the dry season %\citep{condt_2000} 
    \item mean annual precipitation of $2662 \pm 479$ (SD) mm yr$^{(-1)}$ and a 4-month dry season ($<$ 100 mm per month)  
    \item average annual rainfall of about 2640 mm (Detto et al.,
2018) and a well-marked dry season (total rainfall between late-December and mid-April is about 175 mm on average). 
    \item Why is BCI a good choice for this study?
\end{itemize}

\subsubsection*{Prescription of Initial Vegetation Conditions}

\paragraph{BCI 2012 survey data}
After establishment in 1981 in 1981, species and dbh of all living trees $<$ 1 cm have been inventoried every 5 years since 1985 \cite{condit_1995}. 

\paragraph{Soil} 
Soil data?

\subsubsection*{Prescription of Atmospheric Conditions}

\paragraph{Meteorological Data: Choosing which years to run}

I should ask Felicien about CO2 ... 

Taken from Felicien's paper: ... meteorological data from the local flux tower measurements as atmospheric forcings  and used the observed carbon and energy exchange fluxes obtained with the eddy-covariance method to benchmark the modelled productivity and evapotranspiration (Aguilos et al., 2018; Bonal et al., 2008; Powell et al., 2017a). Meteorological data of the simulated years were available at hourly resolution for air temperature, wind speed, specific humidity, precipitation rate, short- and longwave radiation, and were hence used as ecosystem upper boundary condition. To exclude CO2 fertilization effects and keep the same meteorological drivers as in our previous study (di Porcia e Brugnera et al., 2019), the atmospheric concentration of CO2 was fixed at a constant value of 370 ppm, which corresponds to initial concentrations measured by the flux towers.

\subsection*{Ecosystem Demography Model (Version 2.2)}

\paragraph{Model Relevant Processes}

\paragraph{Plant Functional Traits}

\subsection*{Attributing Uncertainty to Ecological Processes}

\subsubsection*{PEcAn}

\subsubsection*{Calculating Parameter Uncertainty}

\paragraph{Meta Analysis}

\begin{itemize}
    \item Which paper should I cite for the hydro traits
    \item BAAD database, employing it's R package
    \item root data from authors I need to add
    \item check for additional data I need to add to data list
    \item BETYdb citation
\end{itemize}

\paragraph{Prior Selection}

I'll need to make a table for this with citation for prior

\begin{itemize}
    \item Parameter
    \item Prior
    \item a
    \item b
    \item Prior meadian
    \item ED2 default
    \item Posterior median
    \item N (sample size)
    \item $\text{CV}_{\text{p,posterior}}/\text{CV}_{\text{p,prior}}$
\end{itemize}

\subsubsection*{Calculating Parameter Sensitivity and Contribution to Model Uncertainty}

\paragraph{Sensitivity Analysis and Variance Decomposition}

\subsubsection*{Benchmarking}

\paragraph{Ensemble Runs}

\newpage


\bibliographystyle{dinat}
\bibliography{Hydro_Papers}
\end{document}